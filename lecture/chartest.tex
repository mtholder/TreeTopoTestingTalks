
%\includepdf[pages={6-18}]{/home/mtholder/Documents/storage/talks/bodega/HolderSotU.pdf} 

\myNewSlide
\section*{We often don't want to test tree topologies}
\begin{itemize}
    \item If we are conducting a ``comparative method'' we have to consider phylogenetic history,
    \item ideally we would integrate out the uncertainty in the phylogeny
\end{itemize}

Tree is a  ``nuisance parameter''

\myNewSlide
\section*{$P$-values with nuisance parameters}
Suppose we are using some positive test statistic, $s$.

If we observe a value of $s=X$ on our data:

Remember that $P = \Pr(s \geq X \mid H_0)$, which
is usually described as the tail probability.
$$P = \int_X^\infty f(s\mid H_0) ds$$

But what if the probability density of $s$ depends on a nuisance
parameter $T$ and we don't know the value of $T$?

\myNewSlide
\section*{$P$-values with nuisance parameters}
$$P = \int_X^\infty f(s\mid H_0) ds$$

We could:
\begin{itemize}
    \item  take the max $P$ over all $T$ -- this is way too conservative
    \item make a confidence $(1-\beta)\%$ set of $T$ and take  $P$ to be $\beta +$ the largest $P$ in that set (Berger and Boos method)
    \item do a Bayesian analysis and get a posterior predictive $P$ value:
    $$ P = \int_X^\infty \left(\int f(s\mid T, H_0)p(T) dT \right)ds $$
\end{itemize}
